\documentclass[a4paper]{article}
    \usepackage{fullpage}
    \usepackage{amsmath}
    \usepackage{amssymb}
    \usepackage{textcomp}
    \usepackage[utf8]{inputenc}
    \usepackage[T1]{fontenc}
    \textheight=10in
    \pagestyle{empty}
    \raggedright

    %\renewcommand{\encodingdefault}{cg}
%\renewcommand{\rmdefault}{lgrcmr}

\def\bull{\vrule height 0.8ex width .7ex depth -.1ex }

% DEFINITIONS FOR RESUME %%%%%%%%%%%%%%%%%%%%%%%

\newcommand{\area} [2] {
    \vspace*{-9pt}
    \begin{verse}
        \textbf{#1}   #2
    \end{verse}
}

\newcommand{\lineunder} {
    \vspace*{-8pt} \\
    \hspace*{-18pt} \hrulefill \\
}

\newcommand{\header} [1] {
    {\hspace*{-18pt}\vspace*{6pt} \textsc{#1}}
    \vspace*{-6pt} \lineunder
}

\newcommand{\employer} [3] {
    { \textbf{#1} (#2)\\ \underline{\textbf{\emph{#3}}}\\  }
}

\newcommand{\contact} [3] {
    \vspace*{-10pt}
    \begin{center}
        {\Huge \scshape {#1}}\\
        #2 \\ #3
    \end{center}
    \vspace*{-8pt}
}

\newenvironment{achievements}{
    \begin{list}
        {$\bullet$}{\topsep 0pt \itemsep -2pt}}{\vspace*{4pt}
    \end{list}
}

\newcommand{\schoolwithcourses} [4] {
    \textbf{#1} #2 $\bullet$ #3\\
    #4 \\
    \vspace*{5pt}
}

\newcommand{\school} [] {
    \textbf{#1} #2 $\bullet$ #3\\
    #4 \\
}
% END RESUME DEFINITIONS %%%%%%%%%%%%%%%%%%%%%%%

    \begin{document}
\vspace*{-40pt}

    

%==== Profile ====%
\vspace*{-10pt}
\begin{center}
	{\Huge \scshape {Cil Satriawan}}\\
	Bandung, Indonesia $\cdot$ cil.satriawan@gmail.com $\cdot$ (62) 821-3163-8069\\
\end{center}

%==== Education ====%
\header{Pendidikan}
\vspace{1mm}
\textbf{Institut Teknologi Bandung}\hfill Bandung, Indonesia\\
Magister Teknik Informatika\hfill Sep 2015 - Jun 2018\\
\vspace{5mm}
\textbf{Institut Teknologi Bandung}\hfill Bandung, Indonesia\\
Sarjana Teknik Informatika\hfill Sep 2008 - Oct 2014\\
\vspace{5mm}
\textbf{National Junior College}\hfill Singapore\\
 \hfill Oct 2003 - Dec 2007\\
\vspace{5mm}

%==== Experience ====%
\header{Pengalaman Kerja}
\vspace{1mm}

\textbf{Prosa} \hfill Bandung, Indonesia\\
\textit{Systems Lead} \hfill Jun 2019 -\\
\vspace{-3mm}
\begin{itemize} \itemsep 1pt
  \item Mengawasi divisi sistem beserta anggotanya
  \item Merancang, membangun, dan mengawasi pengembangan sistem-sistem inti di Prosa
\end{itemize}
\textbf{Prosa} \hfill Bandung, Indonesia\\
\textit{Speech Engineer} \hfill Jan 2018 - Feb 2020\\
\vspace{-3mm}
\begin{itemize} \itemsep 1pt
  \item Merancang dan membangun sistem terkait ucapan, termasuk pengenalan
    ucapan, pengenalan pembicara, dan sintesis ucapan.
  \item Penelitian terkait ucapan, di antaranya termasuk pengumpulan dan
    pengolahan data latih, pengembangan model, serta perancangan percobaan 
\end{itemize}
\textbf{Agranara} \hfill Bandung, Indonesia\\
\textit{Chief Technology Officer} \hfill Dec 2017 -\\
\vspace{-3mm}
\begin{itemize} \itemsep 1pt
  \item Merancang dan membangun berbagai sistem informasi di bidang perbankan
    dan geospasial
  \item Memimpin dan mengelola berbagai tim pengembangan
  \item Merencanakan dan menjalankan strategi jangka pendek dan panjang untuk
    memastikan pertumbuhan dan keuntungan perusahaan
\end{itemize}
\textbf{Freelance Programmer} \hfill Bandung, Indonesia\\
\textit{Software Engineer} \hfill Sep 2016 - Jun 2018\\
\vspace{-3mm}
\begin{itemize} \itemsep 1pt
  \item Mengembangkan perangkat lunak di berbagai bidang, termasuk pengenalan
    ucapan, visualisasi data untuk keperluan perairan, dan sistem informasi
\end{itemize}
\textbf{Institut Teknologi Bandung} \hfill Bandung, Indonesia\\
\textit{Asisten Dosen} \hfill Sep 2015 - Sep 2017\\
\vspace{-3mm}
\begin{itemize} \itemsep 1pt
  \item Pengumpulan dan pengolahan data serta penelitian mengenai pengenalan
    ucapan dan pembicara terotomatisasi dalam rangka membangun sistem transkipsi
    rapat untuk PT INTI
  \item Asisten perkuliahan
\end{itemize}
\textbf{Kuassa} \hfill Bandung, Indonesia\\
\textit{DSP Engineer} \hfill May 2012 - Sep 2015\\
\vspace{-3mm}
\begin{itemize} \itemsep 1pt
  \item Merancang dan membangun perangkat lunak \textit{VST} untuk kebutuhan pengolahan
    audio dan perekaman lagu
  \item Menggunakan berbagai teknik untuk menerjemahkan perangkat keras audio
    analog ke dalam bentuk perangkat lunak
\end{itemize}
\textbf{The British Institute} \hfill Bandung, Indonesia\\
\textit{Instructor} \hfill Oct 2008 - May 2010\\
\vspace{-3mm}
\begin{itemize} \itemsep 1pt
	\item Mengajar bahasa Inggris untuk pembicaraan dan formal
\end{itemize}
\vspace{3mm}

\header{Keterampilan}
\vspace{3mm}
\begin{tabular}{ l l }
\vspace{3mm}
	Bahasa pemrograman: & Python, C, C++, PHP, Go, Java, Javascript \\
	Domain pemrograman: & Scientific programming, Distributed systems, 
  Digital Signal Processing, \\
\vspace{3mm}
                         & Machine Learning, Web development, Information
                         Systems, GIS \\
  Engineering management: & Technology strategy, development synergy \\
\vspace{3mm}
                         & Aligning business, technology, and research
                         concerns \\
\end{tabular}
\vspace{5mm}

\header{Pekerjaan Terpilih}
\vspace{2mm}
{\textbf{BSM Single Dashboard}} \hfill Jakarta, Indonesia \\
{\sl PHP, Javascript} \hfill Jan 2020 - \\
\vspace{1mm}
Sistem informasi berskala nasional yang mengintegrasikan data dan kebutuhan
semua group di bawah direktorat Retail Bank Syariah Mandiri (BSM). Direncanakan
untuk melalui 3 fase pengembangan, dengan fase pertama sedang dalam masa
pengujian.\\
\vspace*{3mm}
{\textbf{Komponen Teknologi Ucapan}} \hfill Bandung, Indonesia \\
{\sl Python, C++} \hfill Oct 2019 - \\
\vspace{1mm}
Komponen terkait teknologi ucapan, termasuk pengenalan ucapan, pengenalan
pembicara, dan sintesis ucapan. Dirancang untuk dapat digunakan oleh aplikasi
lain secara mudah di lingkungan server, termasuk \textit{on-premise} dan di
\textit{cloud}. \\
\vspace*{3mm}
{\textbf{BKN Text To Speech}} \hfill Bandung, Indonesia \\
{\sl Python, C++} \hfill Aug - Oct 2019 \\
\vspace*{1mm}
Sistem sintesis ucapan yang digunakan untuk membangkitkan pertanyaan terucap
untuk siswa tunanetra dalam UAN tingkat SMA, untuk digunakan dalam UAN tahun
2019.\\
\vspace*{3mm}
{\textbf{Strategy & Performance Management System}} \hfill Jakarta, Indonesia \\
{\sl PHP, Javascript, SQL} \hfill Jan - Jun 2019 \\
\vspace*{1mm}
Merancang dan membangun sistem yang memudahkan pengguna group SPM Bank
Mandiri Pusat dalam mendefinisikan dan melakukan visualisasi terhadap data
mereka. Pengguna dapat secara dinamis memuat sumber data, formula, dan
visualisasi data termasuk bar chart, pie chart, bubble chart, dan sebagainya.\\
\vspace*{3mm}
{\textbf{WAIS BSM}} \hfill Jakarta, Indonesia \\
{\sl PHP, Javascript, SQL} \hfill Sep 2018 - \\
\vspace*{1mm}
Implementasi sistem informasi untuk direktorat Wholesale dari Bank Syariah
Mandiri (BSM) yang mengumpulkan dan mengolah data dari \textit{data warehouse}
untuk digunakan di berbagai kebutuhan analisis. Sedang dalam tahap pengembangan
fase 3. \\
\vspace*{3mm}
{\textbf{Voice ID}} \hfill Jakarta, Indonesia \\
{\sl Python, C++, Java} \hfill Dec 2018 - \\
\vspace*{1mm}
Sistem terdistribusi untuk melakukan verifikasi terhadap suara
pelanggan/penelpon pada \textit{call center} Bank Central Asia (BCA) di berbagai
situs. Sistem ini dirancang untuk mengurangi waktu yang diperlukan oleh agen
\textit{call center} untuk memastikan identitas penelponnya. Proyek ini sedang
dalam tahap audit akhir.\\
\vspace*{3mm}
{\textbf{Voice Intel}} \hfill Jakarta, Indonesia \\
{\sl Python, PHP, C++} \hfill May 2018 - Nov 2018 \\
\vspace*{1mm}
Sistem identifikasi dan verifikasi ucapan yang dikembangkan untuk Badan
Intelijen Nasional (BIN). Sistem ini mendaftarkan tersangka melalui rekaman
ucapan yang tersedia secara umum dan mencocokkannya terhadap rekaman. \\
\vspace*{3mm}
{\textbf{SIGI Mandiri}}  \hfill sigi2018.agranara.com\\
{\sl PHP, Javascript, SQL} \hfill Feb 2018 - \\
\vspace*{1mm}
Pengembangan sistem informasi jangka panjang yang digunakan oleh direktorat
Government and Institutional Banking (GVI) Bank Mandiri pusat. Pengembangan
telah melalui 4 fase pengembangan besar, dan masih berlanjut sampai saat ini,
dengan total modul mencapai 47. \\
\vspace*{3mm}
{\textbf{Sistem Pengaduan Konflik Tenurial dan Hutan Adat KLHK}}  \hfill Jakarta, Indonesia\\
{\sl PHP, Javascript, SQL} \hfill 2017 - 2018\\
\vspace*{1mm}
Sistem informasi yang digunakan untuk memetakan dan menyelesaikan masalah
terkait konflik tenurial hutan, terutama hutan adat, antara pemegang kepentingan di
berbagai level. \\
\vspace*{3mm}
{\textbf{Visualisasi Sumur PUSAIR}}  \hfill Bandung, Indonesia\\
{\sl Python} \hfill 2016 - 2016\\
\vspace*{1mm}
Sistem visualisasi 3D yang digunakan untuk menggambarkan titik-titik pengeboran
sumur dan memperkirakan topologi lapisan-lapisan tanah di DKI Jakarta.\\
\vspace*{3mm}
{\textbf{Perisalah INTI/Intens}}  \hfill Bandung, Indonesia\\
{\sl Python, C, C++, Javascript} \hfill 2015 - 2017\\
\vspace*{1mm}
Proyek riset dan pengembangan yang bertujuan mengembangakan sistem pengenalan
ucapan untuk kebutuhan rapat berbagai institusi pemerintahan, terutama MPR/DPR,
bekerja-sama dengan PT INTI melalui pendanaan riset. Hasil akhir adalah sebuah
sistem yg digunakan dalam perangkat Perisalah.\\
\vspace*{3mm}
{\textbf{Kuassa Amplifikation Cerberus}}  \hfill Bandung, Indonesia\\
{\sl C++} \hfill 2014 - 2015\\
\vspace*{1mm}
Perangkat lunak musik yang mensimulasi \textit{amplifier} dan \textit{speaker
cabinet} untuk gitar bas, dengan berbagai model \textit{gain stage} yang
diangkat dari rangkaian listrik merek-merek terkenal, maupun yang dirancang
sendiri. \\
\vspace*{3mm}

\header{Publikasi}
\vspace{2mm}
Satriawan, C.H., Lestari, D.P. (2014). Feature-based Noise Robust Speech Recognition on an Indonesian Language Automatic Speech Recognition System.
Proc. ICEECS.\\
\vspace*{3mm}
Hoesen, D., Satriawan, C.H., Lestari, D.P., Khodra, M.L. (2016). Towards Robust Indonesian Speech Recognition with Spontaneous-speech Adapted Acoustic Models.
SLTU-2016 Workshop on Spoken Language Technologies for under-resourced
languages.\\
\vspace*{3mm}
Satriawan, C. H., Lestari, D.P. (2016). Analyzing and Classifying Indonesian Spontaneous and Dictated Speech. OCOCOSDA 2016.\\
\vspace*{3mm}
Satriawan, C. H., Lestari, D.P. (2018). Average Window Smoothing for an
Indonesian Language Online Speaker Identification System.
International Journal of Electrical Engineering and Informatics.\\

\ 
\end{document}
